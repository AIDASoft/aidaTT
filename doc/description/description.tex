\documentclass[12pt]{article}

\usepackage{a4wide}
\usepackage{graphicx}
\usepackage{url}
\title{A generic tracking toolkit}
\author{Ch. Rosemann, DESY}
\date{\today}



\begin{document}
\maketitle 
\abstract{
In the context of the AIDA European Framework programme blabla a generic tracking toolkit is developed.
It provides the means to the most common functionality needed for tracking in particle physics detectors.  
}

\section{Introduction}
The reconstruction of particle paths that were created in high energetic collisions is one of the central tasks in experimental particle physics.
Since many years both the data acquisition and the data processing are done electronically with computers.
The reconstruction of the paths, the fast and accurate computation of the kinematical properties of the produced particles is the main objective.

The task is usually divided in two subsequent steps, the pattern recognition (or finding) and then the parameter estimation (or fitting).


For the parameter estimation, the established techniques include the effects of material interaction (multiple scattering and energy loss).


\section{Overview}
In general tracking detectors are embedded in a magnetic field.
Charged particles moving through the field follow a helical trajectory, possible distorted by material interactions.
(Multiple) Scattering changes the direction, energy loss changes the curvature due to the Lorentz force.

A track is used as an analogy to describe the particle trajectory.
It consists of a set of 23 parameters -- five to parametrize a helix, 15 values to describe the covariance matrix of the parameters and another three to define the point of reference.

For a tracking toolkit, three different basic modules can be identified that provide individual functionality:
\begin{enumerate}
    \item The propagation: the mathematical expression how the parameters change along the trajectory.
    \item The navigation: the access to the properties of the detector, using the propagation.
    This is for example the information about the measurement surfaces or any material that is crossed by the trajectory.
    \item The fitting: the mathematical procedure to extract the best parameter estimation using the information from the propagation and navigation.
\end{enumerate}



%%%%%%%%%%%%%%%%%%%%%%%%%%%%%%%%%%%%%%%%%%%%%%%%%%%%%%%%%%%%%%%%%%%%%%%%%%%%%%%%
%%%%%%%%%%%%%%%%%%%%%%%%%%%%%%%%%%%%%%%%%%%%%%%%%%%%%%%%%%%%%%%%%%%%%%%%%%%%%%%%


\begin{thebibliography}{99}
\bibitem{aida2.4} \url{http://cds.cern.ch/search?p=AIDA-D2.4}

\bibitem{GBL} C. Kleinwort. General Broken Lines as advanced track fitting method. NIM A673 (2012) 107-110.

%\bibitem{MarlinTPC} MarlinTPC Homepage: https://znwiki3.ifh.de/MarlinTPC/

%\bibitem{millepede} V. Blobel. Software Alignment for Tracking Detectors. NIM A566 (2006) 5-13.

%\bibitem{l3helix}  J. Alcaraz. Helicoidal tracks. L3 Internal Note 1666 (1995)

\bibitem{strandlie}  A. Strandlie, W. Wittek. Derivation of Jacobians for the propagation of covariance matrices of track parameters in homogeneous magnetic fields. NIM A566 (2006), 687-698

\bibitem{geane} M.Innocente,V.Mairie,E.Nagy,GEANE: Average tracking and error propagation package, CERN Program Library,W5013-E,1991.

\bibitem{genfit} C.H\"oppner, S. Neubert, B. Ketzer, S. Paul. A novel generic framework for track fitting in complex detector systems. NIM A620 (2010), 518-525

\end{thebibliography}



\end{document}
